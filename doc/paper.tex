\documentclass[12pt]{article}
\usepackage{amsmath,amsfonts,amssymb}
\usepackage{graphicx}
\usepackage{hyperref}
\usepackage{geometry}
\usepackage{booktabs}
\usepackage[numbers]{natbib}
\geometry{margin=1in}
\usepackage[affil-it]{authblk}
\geometry{margin=1in}

\title{Studying Bitcoin Miners' Strategies Under Uncertainty}
\author{Marius M{\o}ller-Hansen, Enrico Tedeschi}
\affil{UiT - The Arctic University of Norway}
\date{March 2025}

\begin{document}

\maketitle

\begin{abstract}
	In this paper, we apply Bayesian Game Theory to analyze the strategic behavior of Bitcoin miners during periods of high uncertainty, such as volatile price swings and network hash rate fluctuations. We incorporate both endogenous and exogenous parameters to model rational mining behavior and explore equilibrium strategies using simulations.
\end{abstract}

\section{Introduction}
Bitcoin mining plays a crucial role in maintaining the integrity and security of the Bitcoin network, yet it remains highly sensitive to market fluctuations and mining cost dynamics. Recent research has shown that uncertainty in operational costs—particularly electricity prices—and Bitcoin's market value significantly influence miners' strategic decisions. Our prior work \cite{tedeschi2024mining} demonstrates that optimizing for electricity cost reduction can outweigh profit gains from increased transaction fees. Similar incentive-driven mining behavior is explored in the blockchain game-theoretic models of \citeauthor{lewenberg2015inclusive} , and unlike prior work such as \citeauthor{kiayias2016blockchain}, which focuses on mining strategies under complete information or fixed protocols, we introduce a Bayesian game where miners make decisions based on beliefs over uncertain variables like hash rate distribution, electricity costs, and market price.


As our prior work examined the equilibrium between transaction fees and miner profitability, where we proposed a multidimensional profitability model that incorporates fee elasticity, operational costs, and electricity consumption, this paper extends that line of research by introducing a Bayesian Game Theory framework to model how miners form beliefs and adapt strategies during periods of high uncertainty. By incorporating private information—such as individual electricity rates and hardware efficiency—into each miner's type, we provide a comprehensive framework for analyzing decision-making in volatile market conditions. We assume that In a decentralized and competitive ecosystem, miners must continually adjust their strategies to maximize profitability under incomplete information about the network's future state.

% todo: find a problem we are solving by doing so

\section{Background}
Bitcoin mining can be framed as a competitive economic game, where participants expend resources to secure the network and earn rewards in the form of newly minted coins and transaction fees. Traditionally, this has been modeled using classical game theory or economic optimization frameworks \cite{kiayias2016blockchain}, and investigations on miner incentives around transaction inclusion, outlined that under fee pressure and block-space competition, miners deviate from protocol rules in pursuit of higher rewards\,\cite{lewenberg2015inclusive}. We aim to extend in this work a form of strategic adaptation under a Bayesian framework with private types. \citeauthor{kiayias2016blockchain} formalized the mining process as a non-cooperative game, introducing equilibrium strategies where miners choose their computational investment based on expected payoff. However, their model assumes complete information and rationality, making it less suited for capturing belief-based strategies during market volatility.

Bayesian Game Theory provides a natural extension by incorporating uncertainty and private information through the concept of \emph{types}. In this framework, each miner has private information about their own capabilities, constraints, or costs, and forms probabilistic beliefs about those of others. As such, miners optimize their strategies based not only on public information (e.g., network hash rate, Bitcoin price) but also on their beliefs about hidden variables.

Building upon these foundations, our contribution is to offer a formal Bayesian analysis of miner behavior by defining a structured type space and payoff function, followed by simulation-based equilibrium analysis. This allows us to benchmark strategic decisions against volatile parameters, such as hash rate shifts, block reward fluctuations, transaction fee variance, and electricity prices.

\subsection{Bayesian Game Theory}
Game theory provides a structured framework to analyze strategic interactions. In a Bayesian game, players possess private information\textemdash\emph{types}\textemdash and choose actions based on beliefs about others. This is particularly relevant to Bitcoin mining, where miners may not know other miners' costs, hash rates, or strategies.

The key concepts include:
\begin{itemize}
	\item \textbf{Players}: which are the miners.
	\item \textbf{Types}: private information such as hash rate, electricity cost, and risk tolerance.
	\item \textbf{Strategies}: decisions to continue mining, reduce hash rate, switch networks, or stop mining.
	\item \textbf{Payoffs}: determined by rewards (block reward and fees), costs, and relative contribution to total hash rate.
\end{itemize}

\section{Model Setup}
To formalize the miners' behavior under uncertainty, we model the mining process as a Bayesian game. Each miner aims to choose a strategy that maximizes their expected payoff, taking into account both their private information and their beliefs about the strategies and types of other miners.

The decision rule for miner $i$ is given by:
\begin{equation}
	\sigma_i^*(\theta_i) \in \arg\max_{\sigma_i} \mathbb{E}\left[\pi_i(\sigma_i, \sigma_{-i}) \mid \theta_i\right]
\end{equation}

This expression means that the optimal strategy $\sigma_i^*$ for a miner with type $\theta_i$ is the one that maximizes the expected payoff $\pi_i$, given their own strategy $\sigma_i$ and their beliefs over the strategies $\sigma_{-i}$ of the other miners. The expectation is taken with respect to the uncertainty about other miners' types and strategies.

We define the payoff function $\pi_i$ for a miner $i$ as:
\begin{equation}
	\pi_i = \frac{h_i}{H} (R + F)P - C_i
\end{equation}
where:
\begin{itemize}
	\item $h_i$: miner $i$'s individual hash rate (in TH/s)
	\item $H$: total network hash rate (in TH/s)
	\item $R$: block reward (in BTC)
	\item $F$: average transaction fees per block (in BTC)
	\item $P$: current Bitcoin price (in USD/BTC)
	\item $C_i$: miner $i$'s total cost per block (in USD)
\end{itemize}

\subsection*{Example 1: Profitability Estimation}
Let us consider a miner $i$ with the following parameters:
\begin{itemize}
	\item $h_i = 100$ TH/s
	\item $H = 200{,}000$ TH/s
	\item $R = 6.25$ BTC (or $3.125$ BTC now)
	\item $F = 0.75$ BTC
	\item $P = 30{,}000$ USD/BTC (or $\sim 80{,}000$ BTC/USD now)
	\item $C_i = 80$ USD
\end{itemize}
Then the expected payoff per block is:
\begin{equation}
	\pi_i = \frac{100}{200{,}000} \cdot (6.25 + 0.75) \cdot 30{,}000 - 80 = \frac{100}{200{,}000} \cdot 7 \cdot 30{,}000 - 80
\end{equation}
\begin{equation}
	\pi_i = 0.0005 \cdot 210{,}000 - 80 = 105 - 80 = 25 \text{ USD}
\end{equation}
Thus, the miner expects to earn a profit of 25 USD per block mined.

\subsection*{Example 2: Break-even Condition}
Suppose electricity costs increase such that $C_i = 105$ USD. Then:
\begin{equation}
	\pi_i = 105 - 105 = 0 \text{ USD}
\end{equation}
This would be the break-even point. If costs exceed 105 USD, the miner would experience losses and might consider reducing hash power or halting mining.

\subsection*{Miner Type \texorpdfstring{$\theta_i$}{theta}}
In our Bayesian game framework, each miner is characterized by a type $\theta_i$, which captures their private information. This type affects their cost structure, capabilities, and preferences. Possible components of $\theta_i$ include:

\begin{table}[h!]
	\centering
	\renewcommand{\arraystretch}{1.2}
	\begin{tabular}{llll}
		\toprule
		\textbf{Symbol} & \textbf{Description} & \textbf{Nature} & \textbf{Example Value} \\
		\midrule
		$h_i$         & Individual hash rate              & Endogenous & 100 TH/s \\
		$C_i$         & Total operational cost            & Exogenous  & \$100/block \\
		$p_i$         & Electricity price                 & Exogenous  & \$0.05/kWh \\
		$\eta_i$      & Hardware efficiency (J/TH)        & Endogenous & 30 J/TH \\
		$\tau_i$      & Uptime/availability               & Endogenous & 90\% \\
		$\delta_i$    & Discount rate/time preference     & Subjective & 0.95 \\
		$\alpha_i$    & Risk aversion                     & Subjective & Medium \\
		$\kappa_i$    & Taxation/regulatory constraints   & Exogenous  & 10\% tax \\
		$\phi_i$      & Strategic preference              & Endogenous & Join pool \\
		\bottomrule
	\end{tabular}
	\caption{Examples of miner type components $\theta_i$}
\end{table}


These characteristics determine the miner's strategy. For instance, a risk-averse miner with high electricity cost may stop mining during price volatility, while a highly efficient, low-cost miner might scale up operations.

\section{Methodology}
To analyze miners' behavior, we follow these steps:
\begin{enumerate}
	\item \textbf{Model the distribution of types}: define prior distributions for private variables like $C_i$, $h_i$, etc.
	\item \textbf{Simulate belief formation}: miners form beliefs about others' types using public signals and past data.
	\item \textbf{Construct payoff functions}: incorporate endogenous and exogenous parameters.
	\item \textbf{Apply Monte Carlo simulation}: sample from distributions and compute expected payoffs under many scenarios.
	\item \textbf{Identify Bayesian Nash Equilibria (BNE)}: determine best-response strategies for each type.
\end{enumerate}

\section{Monte Carlo Simulation}
Monte Carlo simulation is used to evaluate expected utilities under uncertainty. Random samples of uncertain parameters (e.g., Bitcoin price, total hash rate) are drawn repeatedly to generate distributions of outcomes. This allows us to estimate the likelihood of profitability for various strategies and conditions.

\section{Conclusion and Future Work}
This framework offers a structured method to analyze miner behavior under uncertainty. Future work will involve calibrating the model using real blockchain data, validating miner reactions to past volatility periods, and exploring adaptive strategies in multi-round Bayesian games.

\bibliographystyle{plainnat}
\bibliography{paper}

\end{document}


